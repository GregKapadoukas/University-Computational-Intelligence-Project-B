%! TeX program = lualatex
\documentclass[12pt,a4paper]{article}

\usepackage[nil]{babel}
\usepackage{unicode-math}
\usepackage[svgnames]{xcolor}
\usepackage{lmodern}
\usepackage{graphicx}
\usepackage{wrapfig}
\usepackage{float}
\usepackage{parskip}
\usepackage{xurl}
\usepackage{hyperref}
\usepackage[font=small,labelfont=bf,justification=centering]{caption}

\babelprovide[import=el, main, onchar=ids fonts]{greek} % can also do import=el-polyton
\babelprovide[import, onchar=ids fonts]{english}

\babelfont{rm}
          [Language=Default]{Liberation Sans}
\babelfont[english]{rm}
          [Language=Default]{Liberation Sans}
\babelfont{sf}
          [Language=Default]{Liberation Sans}
\babelfont{tt}
          [Language=Default]{Liberation Sans}

\renewcommand{\thesubsection}{\thesection.\alph{subsection}}
\setlength{\emergencystretch}{3em}

%Enter Title Here
\title{Εργασία Υπολογιστική Νοημοσύνη\\Μέρος Β}
\author{Γρηγόρης Καπαδούκας (ΑΜ: 1072484)}

\begin{document}
\maketitle

\setcounter{section}{-1}
\section{Περιβάλλον Εργασίας - Σύνδεσμος GitHub με Κώδικα}
Για την διεκπεραίωση αυτής της εργασίας έχω επιλέξει να χρησιμοποιήσω γλώσσα προγραμματισμού Python μαζί τις βιβλιοθήκες PyGAD για την υλοποίηση του γενετικού αλγορίθμου και TensorFlow (κυρίως το API της, το Keras) για τον σχεδιασμό και την εκπαίδευση του νευρωνικού δικτύου από το μέρος Α. Επίσης χρησιμοποιώ Pandas, Numpy και Scikit-Learn με σκοπό τον χειρισμό του CSV αρχείου και της προεπεξεργασίας.

Ο κώδικας που γράφτηκε για την εργασία βρίσκεται στο repository στον παρακάτω σύνδεσμο:

\textcolor{blue}{\href{https://github.com/GregKapadoukas/University-Computational-Intelligence-Project-B}{https://github.com/GregKapadoukas/University-Computational-Intelligenc\\e-Project-B}}

Στο repository αυτό συμπεριλαμβάνω και το αρχείο 'environment.yml' το οποίο μπορεί να χρησιμοποιηθεί με χρήση του εργαλείου conda για να δημιουργηθεί πανομοιότυπο Python virtual environment με αυτό που χρησιμοποίησα για τη συγγραφή και εκτέλεση του κώδικα. Βέβαια για χρήση της βιβλιοθήκης TensorFlow απαιτούνται επιπλέον configuration βήματα, που αναλύονται στην εξής σελίδα:
\textcolor{blue}{\href{https://www.tensorflow.org/install/pip}{https://www.tensorflow.org/install/pip}}

\section{Σχεδιασμός Γενετικού Αλγόριθμου}

\subsection{Κωδικοποίηση}

Παρατηρώντας τα δεδομένα του dataset βλέπουμε ότι για οι αισθητήρες είναι συνολικά 4, και τα δεδομένα για κάθε αισθητήρα είναι 3 σε κάθε περίπτωση, δίνοντας συνολικά 12 πραγματικές τιμές στο εύρος [-702, 533] (υποθέτοντας πως οι αισθητήρες παράγουν τιμές στο ίδιο εύρος και πως έχουν παρατηρηθεί τα άκρα των πιθανών τιμών στις μετρήσεις του dataset. Στον κώδικα χρησιμοποιώντας τη συνάρτηση .describe() του Pandas εμφανίζουμε τη μέγιστη και ελάχιστη τιμή για κάθε αισθητήρα που παρατηρήθηκε στο dataset).

Λόγω του συνδυασμού του μεγάλου πλήθους τιμών που χρησιμοποιούνται για την αναπαράσταση της κλάσης κίνησης και του μεγάλου εύρους στο οποίο ανήκουν οι τιμές αυτές, αποφάσισα να χρησιμοποιήσω float-valued encoding, σε συνδυασμό με MinMax scaling στο εύρος [0,1].

Άρα η λύση που προσεγγίζουμε καθώς και κάθε χρωμόσωμα, θα αποτελούνται από ένα πίνακα πραγματικών αριθμών στο εύρος [0,1] μεγέθους 12. Εννοείται πως όταν βρεθεί η βέλτιστη λύση θα κάνουμε inverse MinMax scaling ώστε να επαναφερθούν οι τιμές στο εύρος [-702, 533]. 

\subsection{Πλεονάζουσες Τιμές}

Αρχικά αναφέρω ότι λόγω της χρήσης του float-valued encoding υπάρχει η πιθανότητα να προκύψουν πλεονάζοντες τιμές, δηλαδή τιμές εκτός του εύρους των τιμών των αισθητήρων. Μάλιστα λόγω της χρήσης του MinMax scaling που αναφέρθηκε, το εύρος πλέον στο οποίο πρέπει να περιέχονται όλες οι τιμές των γονιδίων σε κάθε χρωμόσωμα που παράγει ο γενετικός αλγόριθμος είναι το [0,1], στο οποίο με μετέπειτα χρήση inverse scaling θα οδηγήσει στα μη κανονικοποιημένα αποτελέσματα στο εύρος των τιμών του αισθητήρων, ώστε να γίνει η σύγκριση με τους μέσους όρους των τιμών των αισθητήρων για την κλάση 'sitting' από τα δεδομένα του dataset.

Με σκοπό να αποφύγουμε τις πλεονάζουσες τιμές αρκεί να ορίσουμε minimum bound ίσο με 0 και maximum bound ίσο με 1 για τις τιμές που παράγει ο γενετικός αλγόριθμος, με σκοπό αν μετά από μετάλλαξη να παραχθεί τιμή μικρότερη του 0, αυτή να αντικατασταθεί με 0 και αντίστοιχα αν παραχθεί τιμή μεγαλύτερη του 1, αυτή να αντικατασταθεί με 1. 

\subsection{Αρχικός Πληθυσμός}

Για τη δημιουργία του αρχικού πληθυσμού δοκίμασα δύο προσεγγίσεις: 

\begin{itemize}
    \item \textbf{Τυχαία Αρχικοποίηση:} Η πρώτη προσέγγιση είναι να χρησιμοποιήσω τυχαίες τιμές που προέρχονται από ομοιόμορφη κατανομή με εύρος τιμών [0,1]. 

    \item \textbf{Χρήση Latin Hypercube Sampling:} Η δεύτερη προσέγγιση χρησιμοποιεί Latin Hypercube Sampling, όπου αρχικά το εύρος της κάθε τιμής για γονίδιο, δηλαδή το [0,1] σε αυτή τη περίπτωση, χωρίζεται σε όσα μέρη είναι και ο αριθμός των χρωμοσωμάτων στον αρχικό πληθυσμό. Έπειτα για κάθε χρωμόσωμα γίνεται τυχαία επιλογή για κάθε γονίδιο ένα από τα μέρη του πεδίου που προέκυψαν, και διαλέγεται μια τυχαία τιμή από ομοιόμορφη κατανομή στο εύρος αυτό. Η διαδικασία αυτή γίνεται με τρόπο ώστε κάθε μέρος του χωρισμένου πεδίου ορισμού να επιλεχθεί μια φορά για το γονίδιο αυτό στο σύνολο του πληθυσμού.
\end{itemize}

Θεωρητικά η προσέγγιση του Latin Hypercube Sampling είναι η καλύτερη των δύο, επειδή εξασφαλίζει ότι ο αρχικός πληθυσμός θα έχει χρωμοσώματα που θα αναπαριστούν καλύτερα όλο το εύρος πιθανών τιμών, προσθέτοντας βέβαια κάποια επιπλέον πολυπλοκότητα για την αρχικοποίηση.

Στη πράξη όμως δοκιμάζοντας και τις δύο προσεγγίσεις, η διαφορά στα αποτελέσματα για αυτό το πρόβλημα ήταν αμελητέο, αλλά επέλεξα και πάλι το Latin Hypercube Sampling για την θεωρητική ανωτερότητά του σαν προσέγγιση αρχικοποίησης αρχικού πληθυσμού.

Σημειώνω ότι λόγω του εύρους [0,1] που επιλέχθηκε, τα χρωμοσώματα του πληθυσμού είναι ήδη κανονικοποιημένα και μετά από inverse MinMax scaling θα ανήκουν στο εύρος των πιθανών τιμών που θα μπορούσαν να έχουν οι αισθητήρες στο ρομπότ.

\subsection{Υπολογισμός Ομοιότητας}

Αρχικά, με σκοπό το σχολιασμό της καταλληλότητας της ομοιότητας συνημιτόνου σε σχέση με τις άλλες μετρικές ομοιότητας, θα αναφέρω κάποια βασικά ζητήματα για τις μετρικές:

Η ευκλείδεια απόσταση και η απόσταση Manhattan, χρησιμοποιούν την δεύτερη και πρώτη νόρμα αντίστοιχα, με σκοπό να υπολογίσουν την απόσταση. Το προτέρημα της ευκλείδειας απόστασης είναι ότι για μεγαλύτερη διαφορά μεταξύ των τιμών των γονιδίων, η απόσταση γίνεται μεγαλύτερη με τετραγωνικό ρυθμό, οπότε δημιουργείται μεγαλύτερη αύξηση στη τελική απόσταση, σε σχέση με την απόσταση Manhattan, όπου η απόσταση αυξάνεται με γραμμικό ρυθμό όσο αυξάνεται η διαφορά στα γονίδια.

\begin{itemize}
    \item \textbf{Ευκλείδεια και Manhattan Αποστάσεις:} Οπότε αν ήθελα να χρησιμοποιήσω ευκλείδεια ή Manhattan απόσταση ως συνάρτηση καταλληλότητας, θα έπρεπε να αυτό να γίνεται σε μια σχέση όπου όσο μειώνεται η απόσταση κάθε φορά, τόσο να αυξάνεται η συνάρτηση καταλληλότητας, εφόσον οι γενετικοί αλγόριθμοι προσπαθούν να αυξήσουν όσο δυνατό τη συνάρτηση καταλληλότητας ενώ εγώ θέλω να ελαχιστοποιήσω την απόσταση.

    \item \textbf{Ομοιότητα Συνημιτόνου:} Η ομοιότητα συνημιτόνου από την άλλη υπολογίζει τη τιμή του συνημιτόνου της γωνίας που προκύπτουν μεταξύ των δύο διανυσμάτων από τα οποία υπολογίζεται η μετρική. Έχει εύρος πιθανών τιμών από 0 έως 1 (για μη αρνητικά διανύσματα όπως εδώ, αλλιώς από -1 έως 1 αν υπάρχουν και αρνητικά διανύσματα) και για πολλαπλάσια των διανυσμάτων που ελέγχονται δεν αλλάζει η γωνία, οπότε δεν αλλάζει και η απόσταση συνημιτόνου. Αυτή η ιδιότητα ανάλογα το πρόβλημα και τη προσέγγιση της λύσης που εφαρμόζεται, μπορεί να συμβάλλει ή να στερήσει την απόδοση της λύσης.
        
    \item \textbf{Συσχέτιση Pearson:} Η συσχέτιση Pearson υπολογίζει τη συσχέτιση μεταξύ των τιμών του πρώτου γονιδίου σε σχέση με το δεύτερο γονίδιο, δηλαδή υπολογίζει αν παρατηρώντας μια νέα τιμή για το ένα γονίδιο, με πόση ακρίβεια μπορούμε να εξάγουμε συμπεράσματα για την τιμή του δεύτερου γονιδίου. Οι τιμές της συσχέτισης Pearson έχουν εύρος από -1, δηλαδή πλήρη αρνητική συσχέτιση, έως 1, δηλαδή πλήρη θετική συσχέτιση, με τιμή 0 να δείχνει καμία συσχέτιση. Η μετρική αυτή δεν είναι τόσο χρήσιμη σε αυτή τη περίπτωση, επειδή η συσχέτιση των γονιδίων κάθε φορά του γενετικού αλγορίθμου σε σχέση με τους μέσους όρους των τιμών του αισθητήρα δεν παρουσιάζει πληροφορία για τη κλάση κίνησης του ρομπότ.
\end{itemize}

Οπότε καταλήγουμε πως η ομοιότητα συνημιτόνου αυτομάτως είναι πιο κατάλληλη από την συσχέτιση Pearson, και ταυτόχρονα έχει ευκολότερη εφαρμογή στη συνάρτηση καταλληλότητας σε σχέση με την ευκλείδεια και Manhattan απόσταση, λόγω του επιθυμητού χαρακτηριστικού να αυξάνεται όσο πιο όμοια είναι τα δύο διανύσματα. 

Ταυτόχρονα επειδή οι τιμές των αισθητήρων του dataset προέρχονται από επιταχυνσιόμετρα, είναι πολύ πιθανό οι πολλαπλάσιες τιμές να ανήκουν κάθε φορά στην ίδια κλάση, κάτι που κάνει τη χρήση της ομοιότητας συνημιτόνου ακόμα πιο επιθυμητή.

\subsection{Συνάρτηση Καταλληλότητας}

\subsubsection{}

\textbf{Σημείωση:} Όλες οι παρακάτω πράξεις έχουν γίνει με τη προϋπόθεση ότι όλα τα διανύσματα είναι θετικά, οπότε η ομοιότητα συνημιτόνου κινείται στο εύρος [0, 1], εφόσον η προσέγγιση σχεδιασμού του ΓΑ που έχω κάνει μέχρι τώρα δεν οδηγεί ποτέ σε αρνητικά διανύσματα. Αν η προσέγγιση ήταν διαφορετική και υπήρχαν αρνητικά διανύσματα, τότε οι παρακάτω πράξεις δεν θα ίσχυαν.

Η θεωρητική μέγιστη τιμή της F είναι για μέγιστη ομοιότητα συνημιτόνου μεταξύ του $v$ και $t_s$, με μηδενική ομοιότητα μεταξύ του $v$ και όλων των $t_s$. Άρα με αντικατάσταση υπολογίζω:

\begin{center}
    $F(v)_\mathrm{max} = \frac{cos(v,t_s) + c(1-\frac{1}{4}\sum_{i\neq s}cos(v,t_i))}{1+c} \implies$

    $F(v)_\mathrm{max} = \frac{1 + c(1-\frac{1}{4}\cdot0)}{1+c} \implies$

    $F(v)_\mathrm{max} = \frac{1 + c(1-0)}{1+c} \implies$

    $F(v)_\mathrm{max} = \frac{1 + c}{1+c} \implies$

    $F(v)_\mathrm{max} = 1$
\end{center}

Στη πράξη όμως, μέχρι και η τιμή με τη μεγαλύτερη εξ ορισμού καταλληλότητα, δηλαδή το διάνυσμα των μέσων τιμών των τιμών του αισθητήρα για τη κλάση κίνησης 'sitting' δεν έχει μηδενική ομοιότητα συνημιτόνου συγκριτικά με τις άλλες μέσες τιμές για τις άλλες κλάσεις κίνησης. Οπότε η μόνη περίπτωση στη πράξη να έχουμε τιμή $F(v) = 0$ είναι για $v = s$ καθώς και $c = 0$.

Ο υπολογισμός πραγματικής ελάχιστης τιμής στη περίπτωση αυτή είναι εξαιρετικά δύσκολη διαδικασία, και μπορεί μόνο να προσεγγιστεί, οπότε θα υπολογίσω μόνο τη θεωρητική ελάχιστη τιμή, για μηδενική ομοιότητα συνημιτόνου με το $s$ και τέλεια ομοιότητα με όλα τα $t_s$ (προφανώς κάτι που δεν θα συνέβαινε ποτέ στη πράξη). Άρα με αντικατάσταση υπολογίζω:

\begin{center}
    $F(v)_\mathrm{min} = \frac{cos(v,t_s) + c(1-\frac{1}{4}\sum_{i\neq s}cos(v,t_i))}{1+c} \implies$

    $F(v)_\mathrm{min} = \frac{0 + c(1-\frac{1}{4}\cdot1)}{1+c} \implies$

    $F(v)_\mathrm{min} = \frac{\frac{3}{4}\cdot c}{1+c} \implies$

    $F(v, c = 0)_\mathrm{min} = \frac{\frac{3}{4}\cdot 0}{1+0} \implies$

    $F(v)_\mathrm{min} = 0$
\end{center}

Οπότε παρατηρούμε ότι δεν υπάρχουν αρνητικές τιμές καταλληλότητας, εφόσον το θεωρητικό ελάχιστο είναι ίσο με 0. Ο τρόπος που τις αποφύγαμε ήταν μέσω της αποφυγής αρνητικών διανυσμάτων μέσω του MinMax scaling.

\subsubsection{}

Με βάση τις προδιαγραφές της εκφώνησης, η συνάρτηση καταλληλότητας πρέπει να οδηγεί σε υψηλές τιμές όταν τα γονίδια είναι κοντά στις μέσες τιμές των αισθητήρων για τη κατάσταση κίνησης 'sitting' και ταυτόχρονα να οδηγεί σε χαμηλότερες τιμές όταν τα γονίδια είναι κοντά στις μέσες τιμές των αισθητήρων για τις άλλες καταστάσεις. Από την συνάρτηση καταλληλότητας παρατηρούμε τα εξής:

\begin{center}
    $F(v) = \frac{cos(v,t_s) + c(1-\frac{1}{4}\sum_{i\neq s}cos(v,t_i))}{1+c} \implies$

    $F(v) = \frac{cos(v,t_s) + c - c\cdot\frac{1}{4}\sum_{i\neq s}cos(v,t_i)}{1+c} \implies$

    $F(v) = \frac{cos(v,t_s) + c}{1+c} - \frac{c\cdot\frac{1}{4}\sum_{i\neq s}cos(v,t_i)}{1+c} \implies$
\end{center}

Οπότε όσο αυξάνεται το $cos(v,t_s)$, τόσο αυξάνεται το πρώτο φράγμα και το αποτέλεσμα, και όσο αυξάνονται τα $cos(v,t_i)$ τόσο αυξάνεται το δεύτερο φράγμα και μειώνεται το αποτέλεσμα. Η τιμή c ορίζει το ποσοστό συμβολής του δεύτερου φράγματος στο τελικό αποτέλεσμα.

Άρα η συνάρτηση αυτή έχει τα επιθυμητά χαρακτηριστικά και είναι κατάλληλη επιλογή για χρήση ως συνάρτηση καταλληλότητας.

\subsubsection{}

Εφόσον η τιμή της σταθεράς c παίζει το ρόλο ενός ποσοστού που συμβάλλει η ελαχιστοποίηση της ομοιότητας με τις άλλες καταστάσεις στην έξοδο της συνάρτησης καταλληλότητας σε σχέση με την μεγιστοποίηση της ομοιότητας με την κατάσταση 'sitting', θεωρώ ότι ένα κατάλληλο ποσοστό συμβολής είναι 20\%, άρα τιμή $c = 0.2$.

\end{document}
